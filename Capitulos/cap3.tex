\chapter{Investgación Interna}

Para saber más acerca del trato a clientes se tomó la iniciativa de preguntar a las áreas de ventas, servicio al cliente y administradores de comunidades de mercadotecnia , que dudas tienen los clientes, qué estrategias de ventas utilizan para cerrar una venta y cómo es el trato a clientes con dudas específicas.
Ya que la venta es a través del teléfono también se le pidió a los asesores que permitieran oír sus llamadas y así tener de primera mano la estrategia de venta del producto.

\textbf{Ventas}

Se le pidió a un equipo de asesores que tienen trato directo su opinión acerca de cuáles eran las dudas más frecuentes y el interés que tiene la gente en el programa de reparación de crédito. 
Lo que compartieron fue que los clientes tienen dudas sobre varias cosas:

\begin{itemize}
\item ¿Qué es buró de crédito?
\item ¿Cómo funciona la reparación del buró de crédito?
\item ¿Puede demandarlos una institución de crédito?
\item ¿Cuánto pagarían uniéndose al programa?
\item 
\end{itemize}

Oyendo las llamadas a clientes se pueden identificar varias cosas:
Son personas que están desesperadas por salir de deudas
La mayoría tiene deudas bancarias
Tienen casi todos las mismas preguntas y dudas:

\begin{itemize}
\item ¿Sirve el programa?
\item  ¿Es una empresa seria?
\item ¿Si me pueden ayudar a salir de deudas?
\item ¿Cuál es el proceso?
\item ¿Tiene costo?
\item 
\end{itemize}


\textbf{Servicio a Cliente}

Una de las áreas de oportunidad de la reparadora de deuda más grandes es el servicio a cliente. Existe una baja importante en el programa de reparación de crédito. El bot, aunque tiene como objetivo principal la adquisición de clientes, se quiere aprovechar para la retención en el programa.
Hablando con agentes de servicio a cliente y oyendo llamadas de clientes ya en el programa se encontraron varios puntos a resaltar:


\begin{itemize}
\item No entienden por qué no se han liquidado sus deudas
\item Están molestos porque siguen recibiendo llamadas de cobranza
\item No entienden las cuotas que implican estar en el programa
\item No comprenden la importancia de seguir el proceso y las repercusiones de salirse del programa
\item Se les olvida abonar a su cuenta para pagar sus deudas
\end{itemize}


\textbf{Community Managers}

Community managers son las personas encargadas de estar al pendiente de la presencia de la empresa en redes sociales. Entre sus actividades están publicar contenido para atraer a personas a conocer la marca, realizar dinámicas y concursos que den difusión al producto, revisar la reacción de la gente en los perfiles sociales, contestar a comentarios y hablar a través de chat para dar mayor información. El bot toma el lugar de las community manager como herramienta para eficientar el camino hacia la conversión del prospecto en el chat de Facebook Messenger. 
Hablando con la community manager de Resuelve tu deuda y leyendo 100 conversaciones escogidas al azar del año 2016, se observa que las personas interesadas tienen varias dudas:

\begin{itemize}
\item ¿Cómo funciona el programa?
\item ¿Funciona el programa?
\item ¿ A quién han ayudado?
\item ¿Cuales son las condiciones para poder entrar al programa? (monto de deuda, instituciones bancarias, atraso de deuda)
\item ¿Cuanto cobra Resuelve para entrar al programa?
\end{itemize}


\textbf{Estándares}
\begin{itemize}
\item HTTP/S
\item JSON
\end{itemize}


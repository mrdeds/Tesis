\chapter{Pruebas}
\textbf{Metodología}

Analizando los datos de interacción web se sabe que la gente que decide darnos sus datos lo hace el 90\% la primera vez que visita la página(figura 1 del apéndice).Entonces lo más acercado a la realidad sería tener a usuarios nuevos cada vez que se hagan pruebas.
Se realizaron pruebas antes del lanzamiento al público del chatbot. Se eligieron al azar 5 miembros del equipo de marketing, los cuales no tenían conocimiento alguno de la existencia del proyecto. La prueba consistió en decirles que el bot iba a poder atenderlos como si fueran una persona interesada en Resuelve tu deuda y contestar algunas preguntas sobre el producto. Cada individuo participó en una etapa diferente del desarrollo del bot. A continuación se muestran un transcrito de cada iteración de pruebas:

\textbf{Notación:}
\begin{itemize}
\item -  Individuo (“opniones del individuo”)
\item + Bot
\item *  Descripción de eventos en la plataforma
\end{itemize}



\textbf{Pruebas chatbot 1.0}


*Se muestran dos veces el mensaje de bienvenida

-COMO FUNCIONA EL PROGRAMA,

+ "te ayudamos a liquidar deudas que ya no puedes pagar con hasta 70\% de descuento. Nosotros negociamos con tus acreedores y ajustamos tus pagos mensuales a lo que puedas pagar "

- ”Es mucha información”

-como negocian las deudas? con que bancos?

+(Oops, no entendí eso . Intentalo de nuevo ¿Cómo te puedo ayudar?)

-como le hacen para negociar ?

+(Oops, no entendí eso . Intentalo de nuevo ¿Cómo te puedo ayudar?)

-tengo un límite de deuda?

+(Oops, no entendí eso . Intentalo de nuevo ¿Cómo te puedo ayudar?)

-quiero liquidar mis deudas

+¿A quién le debes? (Bancos) (Financieras) (Tiendas departamentales)

-Bancos

+¿A qué banco le debes?

*(AMEX) (BANAMEX) (BANORTE/IXE) (BBVA/BANCOMER) (HSBC) (INBURSA) (INVEX) (SANTANDER) (SCOTIABANK) (BANCO WALMART)

-AMEX

+¿Cuántos meses de atraso tienes?

*(0-5 meses) (6 meses) (7 meses) (8 meses) (9 meses) (10 o más meses)

-6 meses

+¿Cuánto debes en total?

-300000

+¿Cuál es tu correo?

-*****@resuelve.mx

+¿A qué número prefieres que te marquemos?


-1234567890

+Tu deuda de \$ 300000.0 con AMEX con atraso de
Quedaría con un 25 \% de descuento en \$ 225000.0
¡Gracias! Un asesor te contactará pronto

*Menú

\smallskip
\textbf{Observaciones y cambios para siguiente versión}
\smallskip
\textbf{[Etapa]}Principio del camino:


 \textbf{[Cambio]}Que se presente como un bot al principio y establecer expectativas a usuarios: 
Tienes que hablarle o el bot se introduce. 
Se tarda en contestar, establecer esa expectativa en el tiempo de respuesta.  Comunicar: "No es instantáneo, me tardo unos segundos".

\textbf{[Cambio]}
"Hola 'NAME' soy Resuelbot, un robot muy rápido y eficiente creado para ayudarte :), si no tengo una respuesta para tí uno de mis humanos te atenderá. Dime ¿En qué te puedo ayudar?".
\textbf{[Etapa]}Conoce el programa: El plan integral de ahorro no dice mucho.
Te ayudamos a liquidar deudas que ya no puedes pagar con hasta 70\% de descuento. Nosotros negociamos con tus acreedores y ajustamos tus pagos mensuales a lo que puedas pagar ;). 

\textbf{[Cambio]} Los sliders son iguales - quitar el segundo y cambiar el copy a ve nuestro video.

\textbf{[Cambio]} Cotización - se trabó cuando le di un input de texto en la opción múltiple de bancos. 

\textbf{[Cambio]} Ajustar el texto para preguntar “¿Cuál es tu deuda más grande?”

\textbf{[Etapa]}Cotización - no acepta entradas de texto en la opción múltiple de bancos.

\textbf{[Cambio]} Ajustar el texto para preguntar la deuda más grande

\textbf{[Cambio]}Por ahora, ser más claros en el copy añadiendo "no uses comas o signos".

\textbf{[Etapa]}Datos personales:

\textbf{[Cambio]} preguntar “A qué número prefieres que te marquemos?”
[Añadir]Algún horario en particular?

\textbf{[Etapa]}Menú al final de la cotización: 

\textbf{[Cambio]} En lugar de “no sé qué preguntar” cambiar por "Tengo otra pregunta"

\textbf{[Añadir]}Agregar a parte del "Quiero que me hablen" un "Quiero cotizar mi descuento".

\textbf{[Etapa]}Quiero saber más: Necesitamos generar confianza. Hay que dar una respuesta más  convincente: 

\textbf{[Cambio]}Resuelve Tu Deuda fue fundada por Javier y Juan Pablo en 2006, la primera reparadora de crédito en México! emoji: bandera mex y tarjeta -foto-. Hoy somos más de mil colaboradores resolviendo problemas de deuda, hemos ayudado a más de 80k clientes a recuperar su tranquilidad . -foto, footer "mira cuantos somos :o-

\textbf{[Cambio]} “Déjanos tus datos” cambiar por “Quiero una asesoría", tentativamente la gente todavía quiere saber más sobre el producto, suena muy agresiva la primera opción. 

\textbf{[Etapa]}Casos de éxito/testimonios

\textbf{[Cambio]}Hay que tener un copy de transición entre el input que sale de la opción múltiple "Estos son algunos de nuestros clientes y sus historias, conócelos"....Agregar a las opciones del bloque, menú.  

\textbf{[Etapa]}Sucursales

\textbf{[Cambio]}Cambiaría el copy a "encuentra la sucursal que más cerca te quede" o contáctanos por teléfono agregar la opción de teléfono o un botón que llame-> agregamos la cotización. Segunda versión podemos personalizar el bloque de cada zona para que no te lleve a la página general de sucursales. 
Testimonios-> punto 8.

\textbf{[Etapa]} Validación sobre el mínimo de deuda:
Agregar que tienes que estar atrasado y revisar el copy: "Los requisitos para entrar al programa Resuelve son: que estés atrasado en el pago de tus deudas, que tus deudas sumen más de \$35,000 pesos y que esas deudas sean con bancos, tiendas departamentales, Crédito Familiar o Credomatic.

\textbf{[Etapa]} Beneficios: las imágenes de los beneficios son de baja calidad, cambiarlas. " Estos son los beneficios del programa Resuelve:"
Conseguimos descuentos de hasta 70\% ;)
Nosotros nos encargamos de negociar con los bancos .
Atendemos las llamadas de cobranza.
Te damos asesoría legal y financiera gratis a lo largo de todo el programa Resuelve.
Empezamos a sanear tu Buró para que seas candidato a crédito en el futuro.
Total visibilidad del proceso de negociación y de repago a través de nuestra aplicación móvil.

\textbf{[Nueva Etapa]} Después del escenario de éxito hay que agregar un mensaje que pida un comentario para mejorar el bot- Después de cada cotización.

\textbf{[Mejoras]}Métricas. Qué vemos, cómo nos damos cuenta de en qué momento fallamos? 
\textbf{[Nueva opción]} Dar la opción para que pueda contestar un agente de servicio a cliente. 

\medskip
\textbf{Mejoras para la siguiente versión:}

Cambiar copies con la revisión previa
Guardar datos del formulario en una hoja de cálculo de en línea “Google Sheet”
Hacer cálculo de descuento y regresar el cálculo como respuesta

\medskip
\textbf{Pruebas chatbot 2.0}

Se hacen algunas modificaciones al flujo y mensajes del bot para mejorar la experiencia:
\begin{itemize}
\item Se añaden todas las mejoras comentadas en la prueba 1.0 
\item Se añaden algunos mensajes extra para hacer más natural el segundo saludo del bot.
\item Se alimenta a la lista de entradas diferentes saludos para hacer más robusta la interacción
\item Se da formato a los datos que se guardan en la BD

\end{itemize}

Se hicieron algunas pruebas con algunas personas que no estaban involucradas al proyecto pero son parte de la empresa.
Se les dijo que probaran el chatbot sin ninguna instrucción adicional. 
Fueron 10 personas. La mayoría optó por pedir más información acerca de la empresa, calcular el descuento sobre alguna deuda y dejar sus datos. 
Todos llegaron al mensaje por default, el cual indica que lo escrito no fue reconocido por el chatbot como una entrada esperada.
Se pidió proporcionar retroalimentación sobre la experiencia con el chatbot. Se recopilaron varias opiniones:
"Debería poder reconocer diferentes tipos de entradas cuando te pregunta sobre tu deuda"
"Hay algunas cosas que no me supo contestar que la gente usualmente tiene dudas antes de entrar al programa"
"Debería poder responder alguien si no entiende mi pregunta"
Se hacen algunos cambios al flujo para hacer más clara la interacción y limitaciones del chatbot:
Se añade a la lista de comandos la opción de hablar con una persona.
Se añade a la pila de entradas de reconocimiento de lenguaje natural frases como: "cuanto cobran?","
\medskip
\textbf{Pruebas chatbot 3.0}
Ahora se asocia el chatbot con el Fan Page de Resuelve Tu Deuda México y se le asigna a una persona que se encargue de responder a usuarios que pidan hablar con un humano.
Para hacer
Se observaron diversos tipos de interacción. Se agrupan las conversaciones en tres tipos:
-Nueva interacción
-Servicio a cliente
-Empleados (testers)
Con las nuevas interacciones se puede observar que siguen el flujo, van casi directo a pedir una cotización y dejar sus datos para un posterior contacto con un asesor. Se puede notar un nivel de satisfacción positivo
Las personas del grupo de servicio a cliente se le 
Hay que ver como segmentar a la gente que tiene mayor interacción. scoring
Se añade un typing en bloques largos i.e. "Quiero saber más"
Piden préstamos
deudas que no son suyas
cuando no está la Community Manager
Ser más claros con el tener que escoger una opción

\medskip
\textbf{Pruebas chatbot 4.0}
Varios se quedan a la mitad del camino a la hora de dejar sus datos

\medskip
\textbf{Pruebas chatbot 5.0}
Salen 2 ads que dirigen la conversación al bot [pedirle a Diego las imágenes]
Se añade la opción de decir de donde nos vista si no es de México o Colombia y se guarda en otra Hoja de Google Sheets
Se quita el mensaje que indica el monto mínimo si es menor la deuda a \$35,000
